\section{Présentation de Theodo}

\subsection{Activités et croissance}

Theodo est une startup de développement web agile dont le but est de résoudre efficacement les problèmes \foreignlanguage{english}{business} de ses clients.

Elle a été fondée en 2009 par Benoît Charles-Lavauzelle et Fabrice Bernhard, tous deux diplômés de l'Ecole polytechnique et partageant une même passion pour le web. Theodo a multiplié par 10 son chiffre d'affaires ainsi que son effectif en 4 ans pour arriver aujourd'hui à une centaine d'employés et à 13 millions d'euros de chiffre d'affaire.

La stratégie de Theodo est de créer un nouveau métier qui se situe entre les ESN\footnote{\textit{Entreprise de service du numérique}, anciennement \textit{société de services en ingénierie informatique} (SSII ou SS2I).} classiques et les sociétés de conseil. L'objectif vis-à-vis des clients est donc de les conseiller dans le choix d'une solution répondant à une problématique business, mais également de développer cette solution tout en accompagnant la transformation digitale des entreprises.

Pour mener à bien cette double mission dans chacun des projets, Theodo recrute ses développeurs dans les plus grandes écoles d'ingénieur pour s'assurer d'avoir des profils polyvalents, à l'aise techniquement mais également capables de comprendre et de répondre aux problématiques métier des clients. Les développeurs sont accompagnés dans leur progression et l'objectif de Theodo est de former des CTO\footnote{\textit{\foreignlanguage{english}{Chief Technology Officer}}, en français \textit{directeur de la technologie}.} en cinq ans et d'accompagner ceux qui le souhaitent dans la création de leur entreprise dans le cadre de la \textit{\foreignlanguage{english}{Theodo Academy}}, la "startup studio" de Theodo.

\subsection{Une sous section}

Placeholder

\subsection{Une sous section}

Placeholder