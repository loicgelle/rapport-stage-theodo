\section{Présentation de Theodo}

\subsection{Activités et croissance}

Theodo est une startup de développement web agile dont le but est de résoudre efficacement les problèmes \foreignlanguage{english}{business} de ses clients.

Elle a été fondée en 2009 par Benoît Charles-Lavauzelle et Fabrice Bernhard, tous deux diplômés de l'Ecole polytechnique et partageant une même passion pour le web. La stratégie de Theodo est de créer un nouveau métier qui se situe entre les ESN\footnote{\textit{Entreprise de service du numérique}, anciennement \textit{société de services en ingénierie informatique} (SSII ou SS2I).} classiques et les sociétés de conseil. L'objectif vis-à-vis des clients est donc de les conseiller dans le choix d'une solution répondant à une problématique business, mais également de développer cette solution tout en accompagnant la transformation digitale des entreprises.

Pour mener à bien cette double mission dans chacun des projets, Theodo recrute ses développeurs dans les plus grandes écoles d'ingénieur pour s'assurer d'avoir des profils polyvalents, à l'aise techniquement mais également capables de comprendre et de répondre aux problématiques métier des clients. Les développeurs sont accompagnés dans leur progression et l'objectif de Theodo est de former des CTO\footnote{\textit{\foreignlanguage{english}{Chief Technology Officer}}, en français \textit{directeur de la technologie}.} en cinq ans et d'accompagner ceux qui le souhaitent dans la création de leur entreprise dans le cadre de la \textit{\foreignlanguage{english}{Theodo Academy}}, la "startup studio" de Theodo.

Les résultats et la croissance de Theodo sont impressionnants : la jeune entreprise a multiplié par 10 son chiffre d'affaires ainsi que son effectif en 4 ans pour arriver aujourd'hui à une centaine d'employés et à 13 millions d'euros de chiffre d'affaire. La satisfaction des clients, qui est l'unique indicateur de succès de Theodo, est très importante avec un taux de "réachat" supérieur à 100\%. Ces clients sont par ailleurs d'horizons très variés et vont des startups comme LaFourchette ou BlablaCar aux grands comptes comme la BNP, la Société Générale et LVMH. Enfin, l'esprit d'équipe est très fort chez Theodo ; cela se traduit notamment par une grande satisfaction des employés : en 2015, Theodo était la deuxième startup au classement "Happy at work".

\subsection{La méthodologie de gestion de projet comme facteur clé de succès}



\subsection{Organisation de l'entreprise}

Chez Theodo, le principe d'organisation est très simple : pour que tout le monde pousse l'entreprise dans la même direction, chacun doit faire partie d'une équipe projet. De cette manière, tous les employés mettent la méthodologie en pratique au quotidien et peuvent créer de la valeur pour Theodo et pour leurs clients. Même le CEO\footnote{\textit{\foreignlanguage{english}{Chief Executive Officer}}, en français \textit{directeur général}.} et le CTO sont staffés sur des projets !

En conséquence, l'entreprise adopte une organisation "à plat" dans laquelle les tâches de chacun ne sont pas définies par une hiérarchie précise, mais par les projets en cours et les rôles éventuels - voir plus bas. Surtout, pour pouvoir assurer un double service de conseil et de développement auprès de ses clients, Theodo ne compte que deux "profils" d'employés :
\begin{enumerate}
\item{Les \textit{ingénieurs-développeurs}}
\item{Les \textit{\foreignlanguage{english}{business developers}}}
\end{enumerate}

Ce sont à partir de ces deux profils que les équipes projet sont constituées : les développeurs forment l'équipe technique alors que le Scrum Master et le commercial sont des \textit{\foreignlanguage{english}{business developers}}. Les employés sont donc regroupés par équipe projet au sein de l'open space et se déplacent dès qu'ils en changent.

Certains employés obtiennent des rôles qui s'ajoutent à leurs projets dès lors qu'ils ont assez progressé au sein de l'entreprise. Chez les \textit{\foreignlanguage{english}{business developers}} :
\begin{enumerate}
\item{Les administrateurs financiers ont pour rôle de gérer les finances de l'entreprise}
\item{Les membres de l'équipe \textit{\foreignlanguage{english}{sales}} trouvent des nouveaux projets en prospectant parmi les clients potentiels}
\item{Les membres de l'équipe \textit{\foreignlanguage{english}{Joinus}} sont chargés du recrutement}
\end{enumerate}

Chez les développeurs, les architectes sont des développeurs expérimentés qui sont compétents pour réaliser les challenges techniques de leurs projets\footnote{La challenge technique est le moment du choix de la solution technique la plus adaptée au problème \foreignlanguage{english}{business} du client. C'est aussi à ce moment là que les outils de développement - langage et \foreignlanguage{english}{framework} - sont fixés.}. Ils sont staffés sur plusieurs projets simultanément - trois au maximum - et apportent leur expérience pour s'assurer que les projets réussissent ; ils sont des éléments très importants pour la progression de Theodo et la qualité des produits développés.

Un des éléments moteurs de la progression chez Theodo est le système de coachs et de coachés. Chacun a un coach parmi les employés expérimentés, vers lequel il peut se tourner\footnote{Le fait de demander de l'aide à son coach s'appelle le \textit{andon}. Ce terme, issu du système de production Toyota, désigne initialement un signal lumineux qui indique qu'un poste de travail rencontre un problème, provoquant une interruption partielle de la chaîne de production et la résolution immédiate du problème.} dès qu'il a un doute ou un problème - d'ordre technique ou lié à son projet, par exemple.
